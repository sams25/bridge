% Convention sheet

% Anything in \note is optional, and I can write a script to generate an
% annotated and non-annotated version of our conventions.

\documentclass{article}

\usepackage[utf8]{inputenc}
\usepackage{hyperref}
\usepackage[dvipsnames]{xcolor}
\usepackage{parskip}

%Overriding hollow suits
\DeclareSymbolFont{extraup}{U}{zavm}{m}{n}
\DeclareMathSymbol{\varheart}{\mathalpha}{extraup}{86}
\DeclareMathSymbol{\vardiamond}{\mathalpha}{extraup}{87}

% Symbols
\newcommand{\Hs}{\textcolor{Red}{$\varheart$}}
\newcommand{\Ss}{\textcolor{Blue}{$\spadesuit$}}
\newcommand{\Ds}{\textcolor{Orange}{$\vardiamond$}}
\newcommand{\Cs}{\textcolor{Green}{$\clubsuit$}}
\newcommand{\NTs}{\textbf{\footnotesize{NT}}}
% Suits
\renewcommand{\H}[1]{\textcolor{Red}{\textbf{#1}\Hs}}
\renewcommand{\S}[1]{\textcolor{Blue}{\textbf{#1}\Ss}}
  \newcommand{\D}[1]{\textcolor{Orange}{\textbf{#1}\Ds}}
  \newcommand{\C}[1]{\textcolor{Green}{\textbf{#1}\Cs}}
  \newcommand{\NT}[1]{\textbf{#1\NTs}}
  \newcommand{\suits}[1]{\textbf{#1}\Cs\Ds\Hs\Ss}
  \newcommand{\minors}[1]{\textcolor{Brown}{\textbf{#1}}\Cs\Ds}
  \newcommand{\majors}[1]{\textcolor{Purple}{\textbf{#1}}\Hs\Ss}
  \newcommand{\notclubs}[1]{\textbf{#1}\Ds\Hs\Ss}
  \newcommand{\reds}[1]{\textcolor{Red}{\textbf{#1}}\Ds\Hs}

% Annotations
\newcommand{\note}[1]{\textcolor{gray}{#1}}

\title{Bridge Convention}
\author{Anindya Sharma and Freddie Wright}
\date{}

\begin{document}
\maketitle

\section{Introduction}

\subsection{Named conventions that we use}
In short, our system is based on:
\begin{itemize}
    \item ACOL (4 card majors)
    \item Strong \C{2}
    \item Three Weak Twos
    \item Weak\NT~
    \item Stayman and Transfers
    \item Ogust convention
    \item Jacoby \NT{2}
    \item Gerber \C{4} for\NT~ contracts
    \item Blackwood (1430 Roman Key Card) \NT{4} for suited contracts
\end{itemize}

\subsection{Notation}
This is how the suits are displayed. \H \S \D \C.\\
We write \textbf{T} instead of \textbf{10}.\\
This is how a hand is displayed. \H{KJ2} \S{AT98} \D{KQ532} \C{7} (13).\\
\note{These are annotations.}

\section{Hand Analysis}

\textbf{Points.} Ace - 4, King - 3, Queen - 2, Jack - 1.

\textbf{Shape.}\\ No voids, no singletons, at most one doubleton is \textbf{balanced}. Anything else is \textbf{unbalanced}.

\note{The probability of a balanced hand is $\sim 0.47$, and the most probable hand, 4-4-3-2 is balanced.}

\note{Notice that all unbalanced hands have a 5-card suit, except for a 4-4-4-1 split.}

\section{Opening Bids}

\subsection{1 level bids}

\subsubsection{\suits{1}}

\begin{itemize}
\item Balanced, 15-19 HCP.
\item Unbalanced, 12-19 HCP.
\end{itemize}

The above is a simplified summary of 1 level suit bids. The following rules also apply.

\begin{itemize}
\item Always bid your longest suit.
\item If you have two longest suits of equal length, do the following.
\begin{itemize}
\item If there are 4 of each, bid the cheaper suit. 
\item If there are 5 or 6, bid the higher, unless they are hearts and spades.
\end{itemize}
\item If 4441 distribution, do the following.
\begin{itemize}
\item With a red singleton, bid the suit below.
\item With a black singeton, bid the middle suit.
\end{itemize}
\item For unbalanced hands, you may open if you have less than 12 HCP, using \emph{Rule of 20}. If the length of your two longest suits added to your point count is 20 or more, you may make a 1 level opening.
\end{itemize}

\subsubsection{\NT{1}}

\begin{itemize}
\item Balanced, 12-14 HCP.
\end{itemize}

\subsection{2 level bids}

\subsubsection{\C{2}}

\begin{itemize}
\item Balanced, 23-37 HCP.
\item Unbalanced, 20-37 HCP.
\end{itemize}

\subsubsection{\notclubs{2}}

To make a weak 2 level bid, you should have at least 3 HCP in the suit you bid. If you have two 6 card suits bid the major. With both majors, bid the suit with the most HCP. If you have two 7 card suits, remember to count your cards before you look at them.

\begin{itemize}
\item 6-9 HCP, 6 card suit.
\item 4-5 HCP, 7 card suit.
\item 3-2 HCP, 8 card suit.
\end{itemize}

\subsubsection{\NT{2}}

\begin{itemize}
\item Balanced, 20-22 HCP
\end{itemize}

\subsection{3 level bids}

\subsubsection{\suits{3}}

%TODO Look into 'Rule of 3 and 2'
To make a 3 level bid, you should have at least 3 HCP in the suit you bid.

\begin{itemize}
\item 6-9 HCP, 7 card suit
\item 4-5 HCP, 8 card suit
\end{itemize}

\subsubsection{\NT{3}}

\begin{itemize}
\item 7 card minor, headed by AKQ, no other aces or kings.
\end{itemize}

\subsection{4 level bids}

\subsubsection{\minors{4}}

\begin{itemize}
\item Some sort of pre-emptive bid.
\end{itemize}

\subsubsection{\majors{4}}

\begin{itemize}
\item Can make game in a major with minimal help from partner, who might pass an opening at the 1 level. Works as a pre-emptive too.
\end{itemize}

\subsubsection{\NT{4}}

Not used

\subsection{7 level bids}

\begin{itemize}
\item Opener can make grand slam by themselves.
\end{itemize}

\section{Responses to Opener}

\subsection{Responses to 1 level suit openings}

%TODO Consider where to put this bit.
When responding with a suit, the priorities are as follows.
\begin{itemize}
\item Bid the longest suit first
\item With 5 or 6 of each bid the higher ranking suit
\item With multiple 4 card suits, bid them up the line.
\end{itemize}

\subsubsection{Weak responses}

%TODO What about distribution?
If you have 6-9 HCP, you must make a weak response as follows.

\paragraph{After a minor opening}

\begin{itemize}
\item Bid a new 1 level suit
\item Raise opener's minor
\item Bid \NT{1}
\end{itemize}

\paragraph{After a major opening}

\begin{itemize}
\item Raise opener's major
\item Bid \S{1} over \H{1}
\item Bid \NT{1}
\end{itemize}

\paragraph{Shut-out Jump Raises}

%TODO How weak?
If you are weak, but have 5 of opener's major and a singleton or void, you may raise immediately to \majors{4}. The opener should then pass.

\subsubsection{Stronger responses}

%TODO Sort this section out.
If you have 10 or more HCP, you may respond at the 2 level. You should only make a 2 level response if you can't make a 1 level suit response. The same rules about suit preference apply here.

\paragraph{Jacoby \NT{2}}

If the opening bid was \majors{1}, bid \NT{2} if you have at least 4 cards in that suit and 13 points.

\paragraph{Natural \NT{2}}
If the opening bid was \minors{1}, bid \NT{2} if you have 10-12 HCP and cover in the unbid suits.

\subsubsection{Super strong responses}

\paragraph{Jump Shifts}

If you have 16 HCP or more, then you should skip a level of bidding and change suit.

\subsubsection{Over interference}

\paragraph{Negative Doubles}

If the opponents interfere, you have 6 or more HCP and there is no bid you can make, then double.
%TODO Sort this out properly. Agree on something more refined than what we have now.

\subsection{Responses to \NT{1}}

%TODO What about 3 level responses? What about trick Stayman?
The \NT{1} opener is very precise, so the aim of the responses is to find game or to minimise damage. They are as follows, in order of priority.

\begin{itemize}
\item If you have a 5 card major suit, bid \D{2} for hearts, and \H{2} for spades. This is called a transfer bid.
\item If you have a 4 card major suit and 11-28 HCP, bid \C{2}. This is the Stayman convention.
\item With exactly 11 HCP, bid \S{2}.
\item With exactly 12 HCP, bid \NT{2}.
\item With 13-17 HCP, bid \NT{3}.
\item With 18-26 HCP, bid \C{4}.
\item With 27-28 HCP, bid \NT{7}.
\end{itemize}

\subsection{Responses to \C{2}}

\begin{itemize}
\item With 0-8 HCP, bid \D{2}.
\item With 9+ HCP, bid your cheapest longest suit \majors{2} or \minors{3}.
%TODO What should 2NT mean here?
\end{itemize}

\subsection{Responses to \notclubs{2}}

%TODO What does change of suit mean? Should you raise to 4 with 4 card support?
%TODO When to consider slam?
%TODO Is the passing range ok?
\begin{itemize}
\item With 6-9 HCP, and 3 of opener's suit, raise to \notclubs{3}.
\item With 10-14 HCP, pass.
\item With 15-38 HCP, bid \NT{2}. This is a relay bid for the Ogust convention.
\end{itemize}

\subsection{Responses to \NT{2}}

%TODO Decide when we should raise to 3NT?
\begin{itemize}
\item If you have a 5 card major suit, bid \D{3} for hearts, and \H{3} for spades. This is a transfer.
\item If you have a 4 card major suit and 4-20 HCP, bid \C{3}. This is Stayman.
\item With 4-11 HCP, bid \NT{3}.
\item With 12-20 HCP, bid \C{4}.
\end{itemize}

\subsection{Responses to \suits{3}}

%TODO When to consider slam?
%TODO Is the passing range ok?
\begin{itemize}
\item With 6-9 HCP and 3 card support, raise to \suits{4}.
\item With 10-15 HCP, pass.
\item With 16-36 HCP, bid game in partner's suit, or bid \NT{3} if you have all side suits covered and opener was a minor.
\end{itemize}

\subsection{Responses to \NT{3}}

When the responder bids clubs, the opener can always correct to diamonds if necessary.
\begin{itemize}
\item With cover in both majors and a minor, and at least one card in the other minor, pass.
\item With a weak hand and no cover, bid \C{4}.
%TODO How strong?
\item With a strong unbalanced hand, bid \C{5}.
\item With 5 quick tricks including enough aces/kings, bid \C{6}.
\item With 6 quick tricks including 3 aces, bid \NT{7}.
\end{itemize}

\section{Opener's rebids}

\subsection{After an unbalanced \suits{1} opening}

\paragraph{Opener has 12-15 HCP}

\begin{itemize}
\item Raise responder's suit. Except after \C{1}, \D{1}.
\item Bid a new 1 level suit, cheapest first. (Skipping over a suit denies 4 cards.)
\item Rebid \suits{2} with 5 in that suit.
\end{itemize}

\paragraph{Opener has 16-18 HCP}

\begin{itemize}
\item Jump raise responder's suit.
\item Bid a new suit.
\item Rebid \suits{3} with 6 cards in that suit.
\end{itemize}

\paragraph{Opener has 19 HCP}

\begin{itemize}
\item Jump to game in responder's suit.
\item Skip a level and bid a new suit.
\item Jump to game in the opened suit with strong 6 cards or 7 cards in that suit.
\end{itemize}

\subsection{After a balanced \suits{1} opening}

%TODO Look up Ron Klinger's improved way of doing this.

\paragraph{After a \suits{1} response}

\begin{itemize}
\item With 15-16 HCP, bid \NT{1}.
\item With 17-18 HCP, bid \NT{2}.
\item With 19 HCP, bid \NT{3}.
\end{itemize}

\paragraph{After a \NT{1} response}

\begin{itemize}
\item With 15 HCP, pass.
\item With 16-17 HCP, bid \NT{2}.
\item With 18-19 HCP, bid \NT{3}.
\end{itemize}

\paragraph{After a \suits{2} response}

\begin{itemize}
\item Bid \NT{3} or \majors{4} with a fit.
\end{itemize}


\subsection{After a \NT{1} opening}

\paragraph{After a \C{2} response}

\paragraph{After a \reds{2}  response}

\paragraph{After a \S{2} response}

\paragraph{After a \NT{2} response}

\paragraph{After a \suits{3} response}

\paragraph{After a \NT{3} response}

\paragraph{After a \C{4} response} 

\subsection{After a \C{2} opening}

\subsection{After a \notclubs{2} opening}

\subsection{After a \NT{2} opening}

\section{Responder's rebids}

%TODO See page 53 of Basic Bridge

\section{Overcalls}
\section{Responses to Overcall}

\section{Probabilities}
\section{High Card Points vs Game points probabilites}
\section{Vulnerability}
\section{Match Points vs IMPs}
\section{Other}
\end{document}
