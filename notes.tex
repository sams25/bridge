% Convention sheet

% Anything in \note is optional, and I can write a script to generate an
% annotated and non-annotated version of our conventions.

\documentclass{article}

\usepackage[utf8]{inputenc}
\usepackage{hyperref}
\usepackage[dvipsnames]{xcolor}
\usepackage{parskip}

%Overriding hollow suits
\DeclareSymbolFont{extraup}{U}{zavm}{m}{n}
\DeclareMathSymbol{\varheart}{\mathalpha}{extraup}{86}
\DeclareMathSymbol{\vardiamond}{\mathalpha}{extraup}{87}

% Symbols
\newcommand{\Hs}{\textcolor{Red}{$\varheart$}}
\newcommand{\Ss}{\textcolor{Blue}{$\spadesuit$}}
\newcommand{\Ds}{\textcolor{Orange}{$\vardiamond$}}
\newcommand{\Cs}{\textcolor{Green}{$\clubsuit$}}
\newcommand{\NTs}{\textbf{\footnotesize{NT}}}
% Suits
\renewcommand{\H}[1]{\textcolor{Red}{\textbf{#1}\Hs}}
\renewcommand{\S}[1]{\textcolor{Blue}{\textbf{#1}\Ss}}
  \newcommand{\D}[1]{\textcolor{Orange}{\textbf{#1}\Ds}}
  \newcommand{\C}[1]{\textcolor{Green}{\textbf{#1}\Cs}}
  \newcommand{\NT}[1]{\textbf{#1\NTs}}
  \newcommand{\suits}[1]{\textbf{#1}\Cs\Ds\Hs\Ss}
  \newcommand{\minors}[1]{\textcolor{Brown}{\textbf{#1}}\Cs\Ds}
  \newcommand{\majors}[1]{\textcolor{Purple}{\textbf{#1}}\Hs\Ss}
  \newcommand{\notclubs}[1]{\textbf{#1}\Ds\Hs\Ss}

% Annotations
\newcommand{\note}[1]{\textcolor{gray}{#1}}

\title{Bridge Convention}
\author{Anindya Sharma and Freddie Wright}
\date{}

\begin{document}
\maketitle

\section{Introduction}

\subsection{Named conventions that we use}
In short, our system is based on:
\begin{itemize}
    \item ACOL (4 card majors)
    \item Strong \C{2}
    \item Three Weak Twos
    \item Weak\NT~
    \item Stayman and Transfers
    \item Ogust convention
    \item Jacoby \NT{2}
    \item Gerber \C{4} for\NT~ contracts
    \item Blackwood (1430 Roman Key Card) \NT{4} for suited contracts
\end{itemize}

\subsection{Notation}
This is how the suits are displayed. \H \S \D \C.\\
We write \textbf{T} instead of \textbf{10}.\\
This is how a hand is displayed. \H{KJ2} \S{AT98} \D{KQ532} \C{7} (13).\\
\note{These are annotations.}

\section{Hand Analysis}

\textbf{Points.} Ace - 4, King - 3, Queen - 2, Jack - 1.

\textbf{Shape.}\\ No voids, no singletons, at most one doubleton is \textbf{balanced}. Anything else is \textbf{unbalanced}.

\note{The probability of a balanced hand is $\sim 0.47$, and the most probable hand, 4-4-3-2 is balanced.}

\note{Notice that all unbalanced hands have a 5-card suit, except for a 4-4-4-1 split.}

\section{Opening Bids}

\subsection{1 level bids}

\subsubsection{\suits{1}}

\begin{itemize}
\item Balanced, 15-19 HCP.
\item Unbalanced, 12-19 HCP.
\end{itemize}

The above is a simplified summary of 1 level suit bids. The following rules also apply.

\begin{itemize}
\item Always bid your longest suit.
\item If you have two longest suits of equal length, do the following.
\begin{itemize}
\item If there are 4 of each, bid the cheaper suit. 
\item If there are 5 or 6, bid the higher, unless they are hearts and spades.
\end{itemize}
\item If 4441 distribution, do the following.
\begin{itemize}
\item With a red singleton, bid the suit below.
\item With a black singeton, bid the middle suit.
\end{itemize}
\item For unbalanced hands, you may open if you have less than 12 HCP, using \emph{Rule of 20}. If the length of your two longest suits added to your point count is 20 or more, you may make a 1 level opening.
\end{itemize}

\subsubsection{\NT{1}}

\begin{itemize}
\item Balanced, 12-14 HCP.
\end{itemize}

\subsection{2 level bids}

\subsubsection{\C{2}}

\begin{itemize}
\item Balanced, 23-37 HCP.
\item Unbalanced, 20-37 HCP.
\end{itemize}

\subsubsection{\notclubs{2}}

To make a weak 2 level bid, you should have at least 3 HCP in the suit you bid. If you have two 6 card suits bid the major. With both majors, bid the suit with the most HCP. If you have two 7 card suits, remember to count your cards before you look at them.

\begin{itemize}
\item 6-9 HCP, 6 card suit.
\item 4-5 HCP, 7 card suit.
\item 3-2 HCP, 8 card suit.
\end{itemize}

\subsubsection{\NT{2}}

\begin{itemize}
\item Balanced, 20-22 HCP
\end{itemize}

\subsection{3 level bids}

\subsubsection{\suits{3}}

To make a 3 level bid, you should have at least 3 HCP in the suit you bid.

\begin{itemize}
\item 6-9 HCP, 7 card suit
\item 4-5 HCP, 8 card suit
\end{itemize}

\subsubsection{\NT{3}}

Not used

\subsection{4 level bids}

\subsubsection{\minors{4}}

\begin{itemize}
\item Some sort of pre-emptive bid.
\end{itemize}

\subsubsection{\majors{4}}

\begin{itemize}
\item Can make game in a major with minimal help from partner, who might pass an opening at the 1 level. Works as a pre-emptive too.
\end{itemize}

\subsubsection{\NT{4}}

Not used

\subsection{7 level bids}

\begin{itemize}
\item Opener can make grand slam by themselves.
\end{itemize}

\section{Responses to Opener}
\section{Opener's rebids}

\section{Overcalls}
\section{Responses to Overcall}

\section{Probabilities}
\section{High Card Points vs Game points probabilites}
\section{Vulnerability}
\section{Match Points vs IMPs}
\section{Other}
\end{document}
